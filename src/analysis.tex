\section{Analysis}
\subsection{CAPM Analysis}

From the period measured of Jan 01, 2022 to Jan 01, 2024, we see that all assets are less than 1 to the BITW market index, meaning that they are less volatile compared to the assets in the BITW. We expect and see that the cryptocurrencies coin have betas higher than the other asset class measure. The beta is then followed by FAANG stocks like Apple, Amazon, Google. We do see PSP and VNQ within the same beta range of 0.1 - 0.2. 

More over, if we further observe, we see the expected return of all assets not-including cryptocurrencies have a high expected return between 2.5\% - 3.5\%. Cryptocurrencies fall in the range between 0.9\% and 0.22\%, incremental gains compared to the rest of the market.

Overall, we can observe that there is a significant difference in risk and return between regular assets classes like stocks, gold, real estate to cryptocurrencies. More analysis needs to be made to draw further conclusions and longer time horizons need to be observed, but the returns gained from cryptocurrencies stems differently from the returns of other assets, possibly via the different risk exposure in the market.

\subsection{GARCH Model}

- cryptocurrency are more volatile than traditional assets (standard deviation)
- AIC and BIC values are negative, with DJI being the most negative

\subsection{Relative and Absolute Valuation Techniques}

By leveraging on-chain data and relative valuation models like the NVT ratio, investors and analysts can gain a deeper understanding of the intrinsic value and market dynamics of cryptocurrencies. These techniques help in identifying investment opportunities and assessing the fundamental health of different cryptocurrency networks.

In conclusion, the integration of on-chain data and innovative valuation models provides a robust framework for analyzing and valuing cryptocurrencies. As the market for digital assets continues to evolve, these techniques will play a crucial role in guiding investment decisions and fostering a deeper understanding of the complex dynamics at play.


\subsection{NVT Model}

\subsection{Intrinsic Value Model}
-110 and 120
\subsection{LSTM Model}