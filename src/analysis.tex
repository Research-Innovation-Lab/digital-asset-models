\section{Analysis}
\subsection{CAPM Analysis}

From the CAPM (Capital Asset Pricing Model) analysis conducted for the period from January 01, 2022, to January 01, 2024, the following insights can be derived:

Cryptocurrencies exhibit significantly higher betas compared to traditional assets and other alternative asset classes. For instance, BTC-USD and ETH-USD have betas of 0.4680 and 0.5448, respectively, indicating they are more volatile and have higher systematic risk compared to equities, bonds, and commodities. Stocks like TSLA, AAPL, and AMZN also have relatively high betas, reflecting their sensitivity to market movements but generally lower than cryptocurrencies. Alternative assets like VNQ (Real Estate) and PSP (Private Equity) show moderate betas, indicating lower volatility compared to stocks and cryptocurrencies.

The expected returns for cryptocurrencies (e.g., BTC-USD, ETH-USD) are notably lower compared to traditional assets like equities and bonds. This suggests that while cryptocurrencies are more volatile, they do not compensate investors with higher expected returns within this period. AAPL, MSFT, and GOOGL offer higher expected returns compared to cryptocurrencies, aligning with their historical performance in providing growth and income to investors. Assets like DJI (Dow Jones Industrial Average), despite having a very low beta, show a slightly higher expected return, which might indicate stability but lower potential for capital appreciation compared to high-beta assets.

Moving forward, the R-squared values indicate how well each asset's returns are explained by market returns (BITW market index). Higher R-squared values suggest that the asset's returns can be largely attributed to market movements. In addition, the p-values are generally very low across assets, indicating statistical significance in the relationship between their returns and market returns.

The analysis suggests that existing alternative asset investing models, such as CAPM, can be applied to digital assets to some extent, but with notable considerations:

First, cryptocurrencies exhibit significantly higher volatility (measured by beta) compared to traditional assets. This high volatility implies higher risk, which needs to be carefully managed in investment portfolios.
Despite the higher volatility, the results show that cryptocurrencies generally offer lower expected returns compared to equities and other traditional assets over the observed period. This obersation could be due to the fact of the downturn during 2022-2024 period. Given the short period analyzed, longer-term data and analysis are crucial to draw more robust conclusions about the risk-return profiles and applicability of traditional asset models to digital assets.

In conclusion, the CAPM provides us a general basis to for initial insights into the risk and return dynamics of cryptocurrencies and other digital assets. Further research and potentially tailored modeling approaches to fully understand and optimize investment strategies could be applied in addition to the CAPM.

\subsection{GARCH Model}

As shown by the GARCH results, cryptocurrencies exhibit significantly higher volatility compared to traditional assets. This is evident from the standard deviations of residuals, where digital assets like ETH-USD, SOL-USD, and SHIB-USD have noticeably higher values (e.g., ETH-USD: 0.036436, SOL-USD: 0.057214) compared to traditional assets like AAPL (0.015185) and AGG (0.004055). This highlights the inherent volatility and risk associated with digital assets.

In addition, the result provides model fit assessment based on its AIC and BIC. The Akaike Information Criterion (AIC) and Bayesian Information Criterion (BIC) are used to evaluate the goodness of fit of the GARCH model. Lower AIC and BIC values indicate better model fit. In the provided table, both traditional assets and digital assets show negative AIC and BIC values, with DJI (Dow Jones Industrial Average) exhibiting the most negative values among traditional assets. Digital assets like BTC-USD, ETH-USD, and SOL-USD also show strong negative values, indicating that the GARCH model can effectively capture their volatility dynamics.

In addition, the Augmented Dickey-Fuller (ADF) test P-values for the residuals of all assets are extremely low (close to zero), suggesting that the residuals are stationary. This is essential as the GARCH model assumes stationary residuals for accurate volatility forecasting. The high significance of these tests across both traditional and digital assets validates the suitability of the GARCH model in capturing the volatility patterns.

The results suggest that the GARCH model can be applied effectively to digital assets, despite their unique characteristics such as high volatility and limited historical data. GARCH models provide robust tools for managing the heightened volatility of cryptocurrencies. By accurately modeling volatility, investors can implement effective risk management strategies to mitigate potential losses. While GARCH models perform well in capturing digital asset volatility, challenges remain due to the rapidly evolving nature of cryptocurrency markets. Adaptations may be necessary to account for factors like regulatory changes and technological advancements that impact digital asset prices.

\subsection{LSTM Analysis}

The LSTM (Long Short-Term Memory) model has been applied to predict the performance metrics (MSE, RMSE, MAE) for various cryptocurrencies and traditional assets.

Bitcoin shows relatively high errors in MSE (122.7408), RMSE (11.0788), and MAE (9.6324), indicating challenges in predicting its price movements accurately with the LSTM model. Ethereum performs better than Bitcoin with significantly lower MSE (5.1563), RMSE (2.2707), and MAE (1.7916), suggesting the LSTM model captures its price dynamics more effectively. Solana exhibits high MSE (66.9456) and RMSE (8.1820), indicating challenges in accurately predicting its price movements with the LSTM model. Overall, the cryptocurrencies show varying degrees of prediction accuracy. Notably, SHIB-USD has the lowest errors across MSE, RMSE, and MAE, indicating better predictive performance.

For traditional tech stocks like AAPL, MSFT, AMZN, GOOGL, they generally have lower errors in MSE, RMSE, and MAE compared to cryptocurrencies, reflecting the relatively stable and predictable nature of large-cap tech stocks. Other traditional assets like TSLA, SPY, and DJI exhibit varying prediction errors. DJI, in particular, shows extremely high errors, suggesting challenges in predicting its movements with the LSTM model.


Overall, the LSTM models demonstrate varying success in predicting the prices of different cryptocurrencies. Cryptocurrencies like Ethereum and some others show promising results with lower errors, indicating potential applicability of LSTM in forecasting their price movements. High errors observed in cryptocurrencies such as Bitcoin and Solana highlight the volatility and complexity of digital asset markets. Factors such as sudden price fluctuations and lack of historical data can pose challenges for accurate predictions with LSTM models. Traditional assets generally exhibit lower prediction errors compared to cryptocurrencies, reflecting their more stable and mature market characteristics. This suggests that LSTM models may require adaptations or enhancements to effectively handle the unique dynamics of digital assets.

Further research and development are needed to refine LSTM models for digital asset forecasting. Techniques like feature engineering, ensemble methods, and integrating external factors (e.g., sentiment analysis, macroeconomic indicators) could enhance predictive accuracy. While LSTM models show promise in predicting certain cryptocurrencies' prices, their application to digital assets requires careful consideration of the asset's volatility, data quality, and model robustness. Continued advancements in modeling techniques and data availability will be crucial in improving the efficacy of LSTM and other alternative asset investing models for digital assets.