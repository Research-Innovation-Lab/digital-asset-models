\section{Introduction}
\subsection{Background}
Digital assets have grown rapidly in recent years, with the total market cap of cryptocurrencies surpassing \$2 trillion in 2021. While digital assets hold promising investment opportunities, the volatile and decentralized nature of this new asset class presents unique investment challenges compared to traditional assets. As investor interest in digital assets continues to grow, there is a need to explore best practices for evaluating and managing risks associated with digital asset investments.

Alternative asset investing offers strategies that may help address some of the challenges with digital assets. However, applying traditional alternative asset models directly to digital assets is not straightforward given key differences in their characteristics. It is therefore important to examine existing models and understand how they may need to be adapted for the digital asset landscape.

\subsubsection{Alternative Asset Models}
A wide range of alternative asset investing models have been developed and applied across various private market asset classes such as private equity, private credit, real assets and hedge funds. Common models used include portfolio theory, factor models, mean-variance optimization, and risk parity frameworks.

While these models have proven valuable in traditional alternative investments, it remains unclear whether and how they can be applied to digital assets. Digital assets such as cryptocurrencies differ significantly from traditional assets in factors like volatility, liquidity profile and regulatory oversight (FINRA.org).

\subsection {Digital Assets}
Cryptocurrencies were the first widely adopted digital assets, with Bitcoin being the largest. Since then, new digital asset classes have emerged including security tokens representing asset ownership, and utility tokens providing access to a network or platform.

Digital assets present both opportunities and challenges from an investment perspective. While some studies show they can provide portfolio diversification benefits, their short history makes volatility and risk difficult to assess over long periods (FINRA.org). Regulatory ambiguity also introduces legal and compliance complexities for investors (CRS Reports, 2023).

Connecting these concepts, there is a need for research exploring whether and how existing alternative asset investing approaches could benefit the growth of a regulated digital asset investment industry. This paper aims to address this gap by systematically analyzing various models and evaluating their applicability to different digital asset classes.