\subsection{Risk and Return Analysis}

In life, we often here the phrase "There is no free lunch", and that's true in investing. Risk is inherent in all investments. This correlates to the risk-return principle, where the greater the risk the greater the potential return. Investors can only expect higher profit give that they are willing to accept higher chance of losses.

Some risks can be controlled and managed while others cannot. Risk can come in multiple forms: financial risk, market risk. liquidity risk, and inflation risk.  It is important to consider such risk factors when assessing the risk and return of any assets, including digital assets. 

To systematically evaluate the risk and return characteristics of alternative assets, financial models and theories can be employed. A foundation model in finance for assessing the relationship between risk and expected return is the Capital Asset Pricing Model (CAPM).

\subsubsection{CAPM}
The Capital Asset Pricing Model (CAPM) is a cornerstone of modern financial theory, providing a framework for understanding the trade-off between risk and return for individual assets in a diversified portfolio. Developed by William Sharpe in the 1960s \cite{Sharpe1964}, CAPM posits that the expected return of an asset is directly related to its systematic risk, as measured by beta (\(\beta\)). Beta represents the sensitivity of an asset's returns to the returns of the overall market, capturing the asset's exposure to market-wide risk factors.

According to CAPM, the expected return on an asset can be expressed as:
\[
E(R_i) = R_f + \beta_i (E(R_m) - R_f)
\]

where:
\begin{itemize}
    \item[$E(R_i)$] is the expected return of the asset,
    \item[$R_f$] is the risk-free rate,
    \item[$\beta_i$] is the beta of the asset,
    \item[$E(R_m)$] is the expected return of the market,
    \item[$E(R_m) - R_f$] is the market risk premium.
\end{itemize}

The CAPM is based on several key assumptions:
\begin{itemize}
    \item Market Efficiency: All investors have access to all available information and act rationally.
    \item Risk Aversion: Investors are risk-averse, meaning they prefer to minimize risk for a given level of expected return.
    \item Single-Period Investment Horizon: Investors plan for a single period of investment.
    \item Homogeneous Expectations: All investors have the same expectations regarding risk and return.
    \item No Taxes or Transaction Costs: There are no taxes or transaction costs affecting investment decisions.
\end{itemize}

Despite its wide usage, CAPM has several limitations:
\begin{itemize}
    \item Empirical Failures: CAPM does not always accurately predict the relationship between risk and return in real markets.
    \item Simplifying Assumptions: Assumptions such as market efficiency and no taxes are often unrealistic.
    \item Single Factor Model: CAPM considers only systematic risk, ignoring other factors that may affect returns.
\end{itemize}

The CAPM provides a valuable framework for understanding the relationship between risk and return, despite its limitations. Its simplicity and intuitive appeal make it a fundamental tool in finance, but ongoing research and alternative models continue to address its shortcomings.


% Investing in alternative assets offers unique opportunities and challenges, particularly in the realm of risk and return. Unlike traditional assets such as stocks and bonds, alternative assets—including private equity, hedge funds, real estate, and commodities—often exhibit different risk profiles and return characteristics. This distinct behavior necessitates a thorough analysis to understand how these assets can fit into an overall investment strategy and contribute to portfolio diversification and performance.

% When assessing the risk and return of alternative assets, it is crucial to consider factors such as liquidity risk, market risk, and operational risk, which may differ significantly from those associated with traditional investments. Additionally, the return potential of alternative assets can be influenced by factors such as leverage, market inefficiencies, and unique economic exposures.

% To systematically evaluate the risk and return characteristics of alternative assets, financial models and theories can be employed. One of the foundational models in finance for assessing the relationship between risk and expected return is the Capital Asset Pricing Model (CAPM).

