\section{Conclusion}
This research project has explored the applicability of existing alternative asset investing models to digital assets, focusing on the Capital Asset Pricing Model (CAPM), price prediction using Long Short-Term Memory (LSTM) models, and volatility analysis with the Generalized Autoregressive Conditional Heteroskedasticity (GARCH) model. The study aimed to address the fundamental question: "\textit{To what extent can existing alternative asset investing models be applied to digital assets?}"

Abstract and Introduction Revisited

The abstract and introduction set the stage by highlighting the rapid growth and increasing relevance of digital assets, such as cryptocurrencies, in the global financial landscape. They underscored the need for rigorous research and understanding of how traditional asset models could be adapted or augmented to effectively evaluate digital assets. The introduction provided context on the unique characteristics of digital assets, including their volatility, liquidity profiles, and regulatory environment, which pose distinct challenges and opportunities for investors.

CAPM Analysis

The CAPM analysis revealed insightful findings regarding the risk-return tradeoffs of various asset classes, including cryptocurrencies. Cryptocurrencies exhibited significantly higher betas, indicating greater volatility and systemic risk compared to traditional equities, bonds, and commodities. Despite their higher risk profile, cryptocurrencies generally offered lower expected returns during the study period, challenging the conventional wisdom of higher returns associated with higher risk observed in traditional finance. This analysis suggests that while CAPM provides a framework for understanding risk and return relationships, its direct application to digital assets may require adjustments to account for their unique risk characteristics.

LSTM Model for Price Prediction

Using LSTM models for price prediction provided further insights into the predictive capabilities for digital assets. The results showed varying levels of accuracy across different cryptocurrencies, highlighting the potential for machine learning techniques to capture complex market dynamics and inform investment decisions. LSTM models demonstrated their utility in forecasting cryptocurrency prices, although their performance varied depending on the asset and market conditions. This underscores the importance of leveraging advanced computational methods to navigate the volatility and informational inefficiencies inherent in digital asset markets.

GARCH Model for Volatility Analysis

The GARCH model analysis elucidated the volatility dynamics of cryptocurrencies and traditional assets over the study period. Cryptocurrencies exhibited notably higher volatility compared to equities and commodities, underscoring their susceptibility to price fluctuations and market sentiment. The GARCH model's ability to capture time-varying volatility patterns provided valuable insights into risk management strategies for digital asset portfolios. By quantifying and understanding volatility, investors can better assess risk-adjusted returns and tailor their investment strategies accordingly.

Integration and Implications

Integrating findings from CAPM, LSTM price prediction, and GARCH volatility analysis contributes to a comprehensive understanding of digital asset investment dynamics. While traditional asset models like CAPM offer foundational insights into risk and return, they require adaptation to accommodate the unique characteristics of digital assets, such as high volatility and evolving regulatory landscapes. LSTM models enhance predictive accuracy in price forecasting, aiding in portfolio optimization and risk management. The GARCH model's volatility analysis facilitates informed decision-making by identifying periods of heightened risk and potential opportunities.

Conclusion

In conclusion, while existing alternative asset models provide a valuable framework for evaluating digital assets, their direct application requires careful consideration and adaptation. Digital assets present distinct challenges and opportunities that necessitate innovative approaches and continuous research. As the digital asset market matures and regulatory frameworks evolve, further advancements in modeling techniques and empirical research will enhance our ability to effectively integrate digital assets into diversified investment portfolios.

This research underscores the importance of ongoing interdisciplinary collaboration between finance, data science, and regulatory experts to navigate the complexities of digital asset investing and capitalize on emerging opportunities in the global financial landscape.







